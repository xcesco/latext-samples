\documentclass[a4paper,12pt,twoside]{article}

\newcommand\tab[1][100pt]{\hspace*{#1}}

\usepackage{graphicx}
\usepackage[utf8]{inputenc}
\usepackage{lipsum}
\usepackage{geometry}
 \geometry{
 a4paper,
 total={170mm,257mm},
 left=20mm,
 top=20mm,
 }



\author{}
\title{}

\makeatletter
	\newenvironment{indentation}[3]%
	{\par\setlength{\parindent}{#3}
	\setlength{\leftmargin}{#1}       \setlength{\rightmargin}{#2}%
	\advance\linewidth -\leftmargin       \advance\linewidth -\rightmargin%
	\advance\@totalleftmargin\leftmargin  \@setpar{{\@@par}}%
	\parshape 1\@totalleftmargin \linewidth\ignorespaces}{\par}%
\makeatother 

%Path relative to the main .tex file 
\graphicspath{./images}
\begin{document}


\begin{figure}
  \begin{center}
   \includegraphics[width=\textwidth]{./images/logo_atto.jpg}
  \end{center}
\end{figure}

{\raggedright
\begin{indentation}{0pt}{75pt}{0pt}
\textsf{XII LEGISLATURA \indent atti aula}
\end{indentation}
}

\begin{table}
    \begin{tabular}{l{4cm} l}
        Consigliere primo proponente: & ILARIA DAL ZOVO \\
        Consiglieri proponenti: & \begin{flushleft}ANDREA USSAI, TIZIANO CENTIS, ALESSANDRO BASSO, 
        LEONARDO BARBERIO, PIERO MAURO ZANIN, EMANUELE ZANON, GIUSEPPE NICOLI, 
        SIMONA LIGUORI, FRANCO MATTIUSSI, CLAUDIO GIACOMELLI, WALTER ZALUKAR\end{flushleft} \\
    \end{tabular}
\end{table}

\leftline{I’m over on the left.}
\centerline{I’m in the centre.}
\rightline{I’m on the right.}

{\begin{itemize}
    \item 
    What is question 1?
    \item
    What is question 2?
    \item 
    What is question 3?
\end{itemize}}
\parskip = 0pt \parindent = 30 pt
\noindent

\begin{flushleft}
RICORDATO che le norme impugnate consentirebbero lo svolgimento sui prati stabili delle attività autorizzate ai sensi dell'art. 12 della legge regionale 15 ottobre 2009, n. 17 (Disciplina delle concessioni e conferimento di funzioni in materia di demanio idrico regionale), concernente le manifestazioni motoristiche, ciclistiche e nautiche, con o senza mezzi a motore, anche a carattere amatoriale, per l'utilizzo temporaneo di beni del demanio idrico regionale funzionali all'organizzazione e allo svolgimento delle predette manifestazioni;
\end{flushleft}

\begin{flushleft}
Ricordato in particolare che l'argomento dell'impugnazione riguardava i seguenti punti: - l'art. 14 della legge 9/2019 aggiungendo i commi 7-bis e 7-ter all'art. 5 della legge 9/2005 &amp;egrave; intervenuto sulle deroghe alle attività non ammesse sui prati stabili, tra cui la riduzione di superficie, di cui all'art. 4 della medesima legge; - tali disposizioni regolano una specifica procedura per la restituzione in pristino dei luoghi ove si svolgano le manifestazioni autorizzate ai sensi dell'art. 12 della legge 17/2009 qualora il materiale del fondo stradale si depositi accidentalmente sul prato stabile nel corso delle relative attività, individuando nel soggetto organizzatore il responsabile della restituzione in pristino, da effettuarsi entro trenta giorni dal termine dell'attività autorizzata, anche in deroga al divieto di riduzione di superficie; - l'art. 5, comma 1, stabilisce s&amp;igrave; alcune deroghe, ma &amp;laquo;compatibilmente con la disciplina comunitaria e nazionale in materia di conservazione della biodiversità&amp;raquo;;</p>
\end{flushleft}

\begin{flushleft}
RICORDATO quindi che la disposizione impugnata introduce un termine di trenta giorni per la riduzione in pristino dello stato dei luoghi da parte dell'organizzatore e dispone che per tale lasso di tempo non si applichi l'art 4, comma 1, della legge 9/2005 che vieta gli interventi di riduzione di 2 superficie sui prati stabili, introducendo in tal modo una deroga idonea a determinare un abbassamento dei livelli di tutela ambientale sui prati stabili;</p>
\end{flushleft}

\begin{flushleft}
CONSIDERATO che le autorizzazioni per le citate manifestazioni sono sottoposte al parere della struttura 
regionale competente in materia di tutela degli ambienti naturali, qualora il transito interessi SIC e ZPS 
o ricada in aree protette, biotopi e prati stabili e che tale parere è adottato al solo fine di 
accertare che il tracciato non ricada in dette aree, ove dunque deve ritenersi escluso che le attività in 
questione possano essere autorizzate; RILEVATO che la sentenza sottolinea che «pur venendo 
introdotto uno specifico caso di deroga al divieto di riduzione di superficie, resta fermo quanto previsto 
dall'art. 5, comma 1, della legge reg. Friuli-Venezia Giulia n. 9 del 2005, secondo cui tali deroghe 
devono comunque avvenire compatibilmente con la disciplina comunitaria e nazionale in materia di 
conservazione della biodiversità. Di conseguenza, l'attività consentita ai sensi delle norme oggetto
 di censura non potrebbe comunque svolgersi in pregiudizio della disciplina sugli habitat naturali, 
 É essere effettuata nei siti individuati dal d.P.R. n. 357 del 1997» Tutto 
 ciò premesso impegna la Giunta regionale 1) a intervenire sulla procedura di rilascio di 
 autorizzazioni al transito all'interno di aree del demanio idrico regionale per lo svolgimento 
 di manifestazioni motoristiche, ciclistiche e nautiche con o senza mezzi a motore, anche a carattere 
 amatoriale, e a non introdurre alcuna deroga che possa determinare un abbassamento dei livelli di tutela 
 ambientale sui prati stabili.
\end{flushleft}

\begin{flushleft}
Presentata alla Presidenza il giorno \textbf{08/02/2021}.
\end{flushleft}

\end{document}
